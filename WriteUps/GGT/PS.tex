
\documentclass[11pt]{article} 
\usepackage[left=1in, right=1in, top=1in, bottom=1in]{geometry}
\usepackage{amsmath}
\usepackage{amsthm}
\usepackage{amssymb}
\usepackage{amssymb} 
\usepackage{enumitem}
\usepackage{graphicx}
\usepackage[colorlinks=true, linkcolor=blue, citecolor=blue, urlcolor=red]{hyperref}
\usepackage{url}

%\usepackage{quiver}


%\setlist[itemize]{label=(\roman*)}

\theoremstyle{definition}
\newtheorem{definition}{Definition}[section]

\theoremstyle{example}
\newtheorem{example}{Example}[section]

\theoremstyle{remark}
\newtheorem{remark}{Remark}[section]

\theoremstyle{lemma}
\newtheorem{lemma}{Lemma}[section]

\theoremstyle{proposition}
\newtheorem{proposition}{Proposition}[section]

\theoremstyle{Problem}
\newtheorem{problem}{Problem}[section]

\theoremstyle{Solution}
\newtheorem{solution}{Solution}[section]

\theoremstyle{theorem}
\newtheorem{theorem}{Theorem}[section]

\theoremstyle{corollary}
\newtheorem{corollary}{Corollary}[section]

%%\input{commands.tex}

\title{Math 583: Geometric Group Theory}
\author{Problem Sets}
\date{}
\begin{document}
\maketitle


\section{PS1}
\begin{problem}
Let $\mu(G)$ the min cardinality of a generating set of $G$. Show $\mu(G_1*G_2) = \mu(G_1) + \mu(G_2)$.
\end{problem}
\begin{solution}
Let $X=\{g_i\}$, $Y=\{h_i\}$ be generating sets of minimal size. Then, $X\cup Y$ is a generating set for $G_1*G_2$ by definition. $|X\cup Y| = |X| + |Y|$ so we proved $\leq$.\\
Now, let $F \twoheadrightarrow G_1*G_2$ be a presentation with $F$ the free group of rank $\mu(G_1*G_2)$. By Gru\v{s}ko, this surjection decomposes as $F_i \twoheadrightarrow G_i$, so $\mu(G_i) \leq rk(F_i)$. But $F_1*F_2 = F$, so $\mu(G_1)+\mu(G_2) \leq rk(F_1) + rk(F_2) = rk(F) = \mu(G_1*G_2)$. This is the second inequality, so equality has been shown.\\
\end{solution}

\begin{problem}
Let $G$ a f.g. group. Show that for some $n$, $G=G_1*...*G_n$ for $G_i$ indecomposables.
\end{problem}
\begin{solution}
Suppose $G$ is indecomposable. Then we are done. Suppose not, so $G = A*B$. Repeat the argument on $A$ and $B$, and so on. The algorithm is bound to terminate since, by problem 1, the rank of the groups strictly decrease as we go down the tree.\\
\end{solution}

\begin{problem}
Kuro\v{s} theorem: if $H < G_1*G_2$, then $H$ is the free product of free groups and conjugates of subgroups of $G_i$.
\end{problem}
\begin{solution}
Let $X_i$ presentation complexes of $G_i$, $X$ the dumbell with $X_i$. By van Kampen, $\pi_1(X)= G$. Covering spaces of $X$ give subrgoups of $G$. Let $p:(Y,y)\to (X,x)$ a covering space, with $p_*(\pi(Y,y))=H$.
 
Decompose $p^{-1}(X_i)=\cup Y_{ij}$ a disjoint union of sheets. Then, we claim there exists $Z,T$ 1-subcomplexes of $Y$ such that: $y\in Z,T$, $T$ is a tree such that $\pi_1(Y,y)\cong \pi_1(Z,y)\ast \ast \pi_1(Y_{ij} \cup T, y)$. Then, $Z$ is a graph so $\pi_1(Z,y)$ is free. Note that $p_*(\pi_1(Y_{ij}\cup T, y)) = p_*(\pi_1(Y_{ij}, y))$ are conjugate subgroup of $G$.

Construct $T$ in the following way. Choose spanning trees $T_{ij}$ of $Y_{ij}$, $Z=p^{-1}(bar)\cup \cup T_{ij}$. Let $T$ a spanning tree of $Z$ so that $T$ contains all the $T_{ij}$. This can be done by quotienting $Z$ by all $T_ij$ and taking that quotient's spanning tree. The vertices are pulled back to the tree itself. Note that $y$ is in both $Z$ and $T$.

Apply Van Kampen to $\{Z, Y_{ij}\cup T\}$. For that we need all intersections to be trees, which is easy to check. Thus we get the free decomposition of $\pi_1(Y,y)$, and so the proof is complete.\\
\end{solution}

\begin{problem}
Let $G= G_1*G_2$. If $[g,h]\in G_1$ is nontrivial, then $g,h\in G_1$.
\end{problem}
\begin{solution}
Consider $H=\langle g, h\rangle$, which by Kuro\v{s} is $H=(H\cap G_1)*(\text{conjugates of subgroups and free groups})$ where the first term is nontrivial since it contains $[g,h]$. If $H$ is not inside $G_1$, then $C$ is not trivial. By Gru\v{s}ko, the factors are both 1-generated. Then $H\cap G_1$ and $C$ are cyclic (maybe infinite). 

Consider $H\to G\cap G_1$ mapping $C\mapsto 0$. But $H\cap G_1$ is abelian, so $[g,h]\mapsto 0$. But $[g,h]$ was a nontrivial element of $H\cap G_1$, so it should map to something non zero. We have contradicted the assumption that $C$ was non trivial. Thus $H=H\cap G_1$, so we are done.\\
\end{solution}

\begin{problem}
Show that each indecomposable subgroup of $G_1*G_2$ is either $\mathbb{Z}$ or contained in a conjugate of $G_i$.
\end{problem}
\begin{solution}
This follows from Kuro\v{s} theorem.\\
\end{solution}

\begin{problem}
Show that if $G=G_1*G_2$ and for $w\in G$, $w^{-1}G_1w\cap G_i$ is not empty, then $i=1$ and $w\in G_1$.
\end{problem}
\begin{solution}
Let $w^{-1}gw\in G_i$, for some nontrivial $g\in G_1$. Since $w\in G_1*G_2$, we can write $w = w_0 \widetilde{w}$ for $w_0\in G_1$, and $\widetilde{w}$ a reduced word starting with an element of $G_2$. Then, $w^{-1}gw = \widetilde{w}^{-1} g' \widetilde{w}$ for $g' = w_0^{-1}gw_0$. In particular, $g'$ is still in $G_1$.

But $\widetilde{w}^{-1} g' \widetilde{w}$ is an already reduced word inside $G_1$, so no elements of $G_2$ can appear in the expression. Thus, $\widetilde{w} = 1$. Hence, $w = w_0\in G_1$, and so $w^{-1}G_1 w = G_1$. Since $G_1 \cap G_2$ is empty, $w^{-1}G_1w\cap G_i = G_1 \cap G_i$ nontrivial forces $i=1$.\\
\end{solution}

\begin{problem}
Let $G$ be finitely generated. If $G=G_1*...*G_n = H_1*...*H_m$, then $m=n$ and $G_i$ is isomorphic to a conjugate of $H_i$ up to permutation.
\end{problem}
\begin{solution}
If all $G_i$ are $\mathbb{Z}$, then $G$ is free and so the subgroups $H_j$ are also $\mathbb{Z}$. By equality of rank, we obtain the result.

Suppose, after reordering, that $G_i$ is not $\mathbb{Z}$ for $i\leq k$, and $H_j$ is not $\mathbb{Z}$ for $j\leq k'$. Then, $G_1$ is a (not free) subgroup of the free product of the $H_i$, so $G_1<wH_1w^{-1}$ for some $w\in G$ (we choose $H_1$ without loss of generality). Similarily, $H_1 < vG_iv^{-1}$ for some $i$, so we have. Notice $wvG_i(wv)^{-1}\cap G_1$ is nontrivial, so by Q6, we have $wv \in G_1$. Now,
$$G_1 < wvG_i(wv)^{-1} < wH_1w^{-1} < G_1$$
Hence, $G_1$ and $H_1$ are conjugate, thus isomorphic. Apply the same idea to every $G_i$ (up to reordering the $H_i$) for $1\leq i \leq k$, so that every $G_i$ is isomorphic to some $H_j$. We need to check that no two $G_i$ go to the same $H_i$. This is automatic because if it's the case, then $G_i$ is conjugate to $G_j$ and Q6 again, $i=j$.

Note now that $G_1*...*G_k \cong H_1*...H_k$. Letting $G'$ be the normal closure of $G_1*...*G_k$ and $H'$ similarily, we have $G/G'=G/H' = G_{k+1}*...*G_n = H_{k+1}*...*H_m$. This forces $H_i$ to be $\mathbb{Z}$ for $k+1\leq i \leq m$, and by a rank argument, $n-k = m-k$, so $n=m$.\\
\end{solution}

\newpage
\section{PS2}
\begin{problem}
Describe the Bass-Serre tree of $\mathbb{Z}*\mathbb{Z}$.
\end{problem}
\begin{solution}
Infinite-valent tree.\\
\end{solution}

\begin{problem}
Has been replaced.
\end{problem}
\begin{solution}
\end{solution}


\begin{problem}
Show $\pi_1(X) = G_1\ast_{G_0}$, where $X$ is the HNN construction.
\end{problem}
\begin{solution}
Recall $G_1\ast_{G_0} = <G_1, t | t\varphi^2(g)t^{-1} = \varphi^3(g)>$.\\
Consider the cover of $X$ given by: $\{X_1$, a half of the handle and a line from the top of the half handle to the basepoint$\}$ and $\{$ the other half handle $\}$. Now,
$$\pi_1(A,y) = \pi_1(X_1)*\mathbb{Z}$$
$$\pi_1(B,y_0) = \pi_1(X_0)$$
$$\pi_1(A\cap B,y_0) = \pi_1(X_0) * \pi_1(X_0)$$
\end{solution}

\begin{problem}
Find a nontrivial isometric action of $G_1\ast_{G_0}$ on $\mathbb{R}$.
\end{problem}
\begin{solution}
Define $f:G_1\ast_{G_0}\to \mathbb{Z}$ by $f(G_1)=0$ and $f(t)=1$. This extends to a homomorphism of the HNN extension. Since $\mathbb{Z}$ acts nontrivially and isometrically, we are done.\\
\end{solution}

\begin{problem}
FInd a tree with HNN action so that vertex stabilizers are subgroups conjugate to $G_1$ and edge stabilizers are subgroups conjugate to $G_0$.
\end{problem}
\begin{solution}
Collapsing to a tree of the universal cover of the handle construction.
\end{solution}



\end{document}