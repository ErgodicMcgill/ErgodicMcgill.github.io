\documentclass[reqno, twoside]{article}
\input{preamble.sty}
\input{macros.sty}

\newcommand{\Clim}{\operatorname{C-lim}}
\newcommand{\Csup}{\operatorname{C-sup}}
\newcommand{\Dlim}{\operatorname{D-lim}}

\begin{document}
    \title{\textbf{\normalsize\MakeUppercase{Summer 2025 Reading Group on Ergodic Theory}}}
    \author{\normalsize\textsc{Exercise Sheet 9 (Zhaoshen Zhai): Weak mixing and Almost-periodic Functions}}
    \date{}
    \maketitle
    \freefootnote{\textit{Date}: \today.}

    The purpose of this exercise sheet is to give another proof of the following theorem.

    \begin{mainTheorem}\label{theorem}
        If a measure-preserving dynamical system is not weak-mixing, then there exists an almost periodic function that is not constant a.e.
    \end{mainTheorem}

    First, a technical lemma needed for Exercise \ref{equiv}.

    \begin{exercise}\label{tech}
        Let $x_0,x_1,\ldots\in\R$. If $\Clim x_n=x$ and $\Clim x_n^2=x^2$ for some $x\in\R$, then $\Dlim x_n=x$.

        \textsc{Hint:} Show that $\Clim|x_n-x|^2=0$. Why does this imply that $\Dlim x_n=x$?
    \end{exercise}

    The \textit{product} of systems $(X,\mc{B},\mu,T)$ and $(Y,\mc{C},\nu,S)$ is the system $(X\times Y,\mc{B}\otimes\mc{C},\mu\times\nu,T\times S)$, where $\mc{B}\otimes\mc{C}$ is the $\sigma$-algebra generated by $B\times C$ for $B\in\mc{B}$ and $C\in\mc{C}$, and $\mu\times\nu$ is the (unique) measure on $\mc{B}\otimes\mc{C}$ such that $(\mu\times\nu)(B\times C)=\mu(B)\nu(C)$ for all $B\in\mc{B}$ and $C\in\mc{C}$. You can use the following facts.

    \begin{fact*}
        For $f\in L^2(X)$ and $g\in L^2(Y)$, let $f\otimes g:X\times Y\to\R$ by defined by $(f\otimes g)(x,y)\coloneqq f(x)g(y)$. Then
        \vspace{-0.05in}
        \begin{equation*}
            \l\{f\otimes g\st f\in L^2(X),g\in L^2(Y)\r\}
            \vspace{-0.05in}
        \end{equation*}
        linearly span a dense subset of $L^2(X\times Y)$.
    \end{fact*}

    \begin{fact*}[Fubini-Tonelli]
        For $f\in L^2(X)$ and $g\in L^2(Y)$, we have $\int_{X\times Y}f\otimes g\,\d(\mu\times\nu)=\int_Xf\,\d\mu\int_Yg\,\d\nu$.
    \end{fact*}

    \begin{exercise}\label{equiv}
        The following are equivalent for a measure-preserving dynamical system $X$.
        \begin{enumerate}
            \item $X$ is weak mixing.
                \vspace{-0.05in}
            \item $X\times X$ is weak mixing.
                \vspace{-0.05in}
            \item $X\times X$ is ergodic.
        \end{enumerate}
        \textsc{Hint:} Compute $\l\langle(T\times T)^n(f_1\otimes f_2),g_1\otimes g_2\r\rangle$ and use $\E(f_1\otimes f_2)=\E(f_1)\E(f_2)$.
    \end{exercise}

    \begin{exercise}
        Follow the steps below to prove Theorem \ref{theorem}: if a measure-preserving dynamical system $X$ is not weak-mixing, then there exists an almost periodic function that is not constant a.e.
        \begin{enumerate}
            \item Note that we can assume that $X$ is ergodic (no need for ergodic decomposition).
                \vspace{-0.05in}
            \item Show that there is a non-constant $(T\times T)$-invariant function $K\in L^2(X\times X)$. Assume that $\E(K)=0$.
                \vspace{-0.2in}
            \item Show that the Hilbert-Schmidt operator $\Phi_K$ on $L^2(X)$ given by $\Phi_Kf(y)\coloneqq\int_XK(x,y)f(x)\,\d\mu(x)$ (we say that $\Phi_K$ has \textit{kernel} $K$) commutes with $T$. Thus $\Phi_Kf\in\mc{AP}(X)$ for every $f\in L^2(X)$, so it suffices to find one such that $\Phi_Kf\neq0$ (since $\E(\Phi_Kf)=0$).
                \vspace{-0.05in}
            \item Suppose that there is no $f\in L^2(X)$ such that $\Phi_Kf\neq0$. Show that $K$ is orthogonal to $f\otimes g$ for all $f,g\in L^2(X)$, hence $K=0$, a contradiction. \textsc{Hint}: The map $x\mapsto\int_XK(x,y)\,\d\mu(y)$ is $T$-invariant.
        \end{enumerate}
    \end{exercise}
\end{document}
