\documentclass[reqno, twoside]{article}
\input{preamble.sty}
\input{macros.sty}

\newcommand{\Clim}{\operatorname{C-lim}}
\newcommand{\Csup}{\operatorname{C-sup}}
\newcommand{\Dlim}{\operatorname{D-lim}}

\begin{document}
    \title{\textbf{\normalsize\MakeUppercase{Summer 2025 Reading Group on Ergodic Theory}}}
    \author{\normalsize\textsc{Exercise Sheet 8 (Zhaoshen Zhai): Correspondence principles;}\\\normalsize\textsc{The finite-infinite bridge; Baby steps towards the Structure Theorem.}}
    \date{}
    \maketitle
    \freefootnote{\textit{Date}: \today.}

    The first three exercises explore further correspondences between finite Ramsey theory, infinite Ramsey theory, and dynamical systems; you don't need to do all of them. The next few exercises are general lemmas used to study the structure of weak mixing and compact systems.

    Throughout, we let $(X,\mc{B},\mu,T)$ be an invertible measure-preserving dynamical system, and recall that a \textit{topological dynamical} system is a pair $(X,T)$ consisting of a compact metrizable space $X$ and a homeomorphism $T:X\to X$. The invertibility conditions are not necessary, but are here for convenience.

    \begin{exercise}
        Show that the following are equivalent.
        \begin{center}
            \begin{minipage}{0.95\textwidth}
                \begin{theorem*}[Simple recurrence in open covers]
                    Let $(U_\alpha)_\alpha$ be an open cover of a topological dynamical system $(X,T)$. There exists $\alpha$ such that $U_\alpha\cap T^{-n}U_\alpha\neq\e$ for infinitely many $n\in\N$.
                \end{theorem*}
                \begin{theorem*}[Infinite pigeonhole-principle]
                    For any $c\geq1$, any $c$-colouring of $\Z$ always contains a colour class with infinitely-many elements.
                \end{theorem*}
                \begin{theorem*}[Finite pigeonhole-principle]
                    For any $c,k\geq1$, there exists $N(c,k)$ such that if $n\geq N(c,k)$ and we colour $\l\{1,\dots,n\r\}$ by $c$ colours, then there is a colour class of at least $k$ elements.
                \end{theorem*}
            \end{minipage}
        \end{center}
    \end{exercise}

    \begin{exercise}
        Show that the following are equivalent.
        \begin{center}
            \begin{minipage}{0.95\textwidth}
                \begin{theorem*}[Multiple recurrence in open covers]
                    Let $(U_\alpha)_\alpha$ be an open cover of a topological dynamical system $(X,T)$. There exists $\alpha$ such that for each $k\geq1$, we have $\bigcap_{i<k}T^{-in}U_\alpha\neq\e$ for some $n\in\N$.
                \end{theorem*}
                \begin{theorem*}[Infinitary van der Waerden]
                    For any $c\geq1$, any $c$-colouring of $\Z$ always contains a colour with arbitrarily long arithmetic progressions.
                \end{theorem*}
                \begin{theorem*}[Finitary van der Waerden]
                    For any $c,k\geq1$, there exists $N(c,k)$ such that if $n\geq N(c,k)$ and we colour $\l\{1,\dots,n\r\}$ with $c$ colours, then there is a monochromatic $k$-term arithmetic progression.
                \end{theorem*}
            \end{minipage}
        \end{center}
    \end{exercise}

    \begin{exercise}
        Show that the following are equivalent.
        \begin{center}
            \begin{minipage}{0.95\textwidth}
                \begin{theorem*}[Furstenberg's multiple recurrence; v1]
                    For any $k\geq1$ and any set $A\in\mc{B}$ with positive measure, there exists $n\geq1$ such that $\mu(\bigcap_{i<k}T^{-in}A)>0$.
                \end{theorem*}
                \begin{theorem*}[Furstenberg's multiple recurrence; v2]
                    For any $k\geq1$ and any set $A\in\mc{B}$ with positive measure, we have $\liminf_{N\to\infty}\frac{1}{N}\sum_{n<N}\mu(\bigcap_{i<k}T^{-in}A)>0$
                \end{theorem*}
                \begin{theorem*}[Furstenberg's multiple recurrence; v3]
                    For any $k\geq1$ and any $f\in L^\infty(X,\mu)$ with $f\geq0$ and $\int f\d\mu>0$, we have $\liminf_{N\to\infty}\frac{1}{N}\sum_{n<N}\int\prod_{i<k}T^{-in}f\,\d\mu>0$.
                \end{theorem*}
                \begin{theorem*}[Infinitary Szemerédi]
                    Any subset of $\Z$ with positive upper density contains arbitrarily long arithmetic progressions.
                \end{theorem*}
                \begin{theorem*}[Finitary Szemerédi]
                    For any $\delta>0$ and $k\geq1$, there exists $N(\delta,k)$ such that if $n\geq N(\delta,k)$, then any subset of $\l\{1,\dots,n\r\}$ with at least $\delta n$ elements contains a $k$-term arithmetic progression.
                \end{theorem*}
            \end{minipage}
        \end{center}
    \end{exercise}

    \begin{remark*}
        Your proofs of `infinitary $\Leftrightarrow$ finitary' probably use some sort of compactness-and-contradiction argument. One can instead use the Compactness Theorem in first-order logic to establish these results.
    \end{remark*}

    \begin{definition*}
        Let $(v_n)$ be a sequence in a Hilbert space $H$ and let $v\in H$.
        \begin{enumerate}
            \item We say that \textit{$v_n$ converges to $v$ in density}, and write $\Dlim_nv_n=v$, if $\l\{n\in\N\st\l\|v_n-v\r\|>\epsilon\r\}$ has zero upper density for each $\epsilon>0$.
                \vspace{-0.05in}
            \item We say that \textit{$v_n$ converges to $v$ in the Cesàro sense}, and write $\Clim_nv_n=v$, if $\lim_N\frac{1}{N}\sum_{n<N}v_n=v$.
                \vspace{-0.05in}
            \item The \textit{Cesàro supremum} of $v_n$ is $\Csup v_n\coloneqq\limsup_N\l\|\frac{1}{N}\sum_{n<N}v_n\r\|$.
        \end{enumerate}
    \end{definition*}

    \begin{exercise}\label{equiv}
        Let $(v_n)_n$ be a bounded sequence in a Hilbert space $H$ and let $v\in H$. 
        \begin{enumerate}
            \item[a)] Prove that $\Clim_nv_n=0$ iff $\Csup_nv_n=0$.
                \vspace{-0.05in}
            \item[b)] Prove that $\Clim_n\l\|v_n-v\r\|=0$ iff $\Dlim_nv_n=v$.
                \vspace{-0.05in}
            \item[c)] Prove that $\lim_nv_n=v$ implies $\Dlim_nv_n=v$, which in turn implies $\Clim v_n=v$.
                \vspace{-0.05in}
            \item[d)] What happens if $(v_n)$ is unbounded?
        \end{enumerate}
    \end{exercise}

    \begin{definition*}
        A measure-preserving dynamical system $(X,\mu,T)$ is said to be
        \begin{enumerate}
            \item \textit{mixing} if $\lim_n\l\langle T^{-n}f,g\r\rangle=\E(f)\E(g)$ for every $f,g\in L^2(X)$.
                \vspace{-0.05in}
            \item \textit{weak mixing} if $\Dlim_n\l\langle T^{-n}f,g\r\rangle=\E(f)\E(g)$ for every $f,g\in L^2(X)$.
        \end{enumerate}
    \end{definition*}

    \begin{exercise}
        Prove that $(X,\mu,T)$ is ergodic iff $\Clim_n\l\langle T^{-n}f,g\r\rangle=\E(f)\E(g)$ for all $f,g\in L^2(X)$.
    \end{exercise}

    \begin{exercise}[van der Corput Lemma]\label{van}
        Let $(v_n)_n$ be a bounded sequence in a Hilbert space $H$. Prove that if $\Csup_h\Csup_n\l\langle v_n,v_{n+h}\r\rangle=0$, then $\Clim_nv_n=0$.

        \textsc{Sketch}: Normalize $(v_n)$ and prove, via a telescoping estimate and averaging over $\l\{0,\dots,H-1\r\}$, that
        \vspace{-0.05in}
        \begin{equation*}
            \raisebox{-2pt}{$\Biggl\|\,$}\frac{1}{N}\sum_{n<N}v_n\raisebox{-2pt}{$\Biggr\|$}^2\leq O\raisebox{-2pt}{$\Biggl(\!$}\frac{1}{H^2}\sum_{h,h'<H}\frac{1}{N}\sum_{n<N}\l\langle v_{n+h},v_{n+h'}\r\rangle\!\raisebox{-2pt}{$\Biggr)$}+O\l(\frac{H^2}{N^2}\r)
            \vspace{-0.05in}
        \end{equation*}
        for any $N,H\geq1$. Use another telescoping argument to show that
        \vspace{-0.05in}
        \begin{equation*}
            \Csup_n\l\langle v_{n+h},v_{n+h'}\r\rangle=\Csup_n\langle v_{n+|h-h'|},v_n\rangle
            \vspace{-0.05in}
        \end{equation*}
        for any $h,h'<H$, and use this to eliminate $h'$.
    \end{exercise}

    \begin{definition*}
        A function $f\in L^2(X)$ is \textit{weak mixing} if $\Dlim_n\l\langle T^{-n}f,f\r\rangle=0$.
    \end{definition*}

    \begin{exercise}
        Show that for any weak mixing $f\in L^2(X)$, we have $\Dlim_n\l\langle T^{-n}f,g\r\rangle=0$ for every $g\in L^2(X)$. Deduce that $X$ is weak mixing iff every $f\in L^2(X)$ with mean zero is weak mixing.

        \textsc{Hint:} Use Exercises \ref{equiv} and \ref{van}, and Cauchy-Schwarz.
    \end{exercise}

    \begin{exercise}
        For any $f\in L^2(X)$, prove that $\l\{T^nf\st n\in\Z\r\}$ has compact closure in $L^2(X)$ iff for each $\epsilon>0$, the set $\l\{n\in\Z\st\l\|f-T^nf\r\|<\epsilon\r\}$ is syndetic. If any of these hold, we say that $f$ is \textit{almost periodic}.
    \end{exercise}

    \begin{exercise}
        Prove that if $f$ is weak mixing and $g$ is almost periodic, then $\l\langle f,g\r\rangle=0$.
    \end{exercise}

    \begin{exercise}
        Prove that the set $\mc{AP}(X)$ of almost periodic functions in $L^2(X)$ is closed.
    \end{exercise}
\end{document}
