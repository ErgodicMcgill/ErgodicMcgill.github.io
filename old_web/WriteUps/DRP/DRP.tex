\documentclass[11pt]{article} 
\usepackage[left=1in, right=1in, top=1in, bottom=1in]{geometry}
\usepackage{amsmath}
\usepackage{amsthm}
\usepackage{amssymb}
\usepackage{amssymb} 
\usepackage{enumitem}
\usepackage{graphicx}
\usepackage[colorlinks=true, linkcolor=blue, citecolor=blue, urlcolor=red]{hyperref}
\usepackage{url}

%\usepackage{quiver}


%\setlist[itemize]{label=(\roman*)}

\theoremstyle{definition}
\newtheorem{definition}{Definition}[section]

\theoremstyle{example}
\newtheorem{example}{Example}[section]

\theoremstyle{remark}
\newtheorem{remark}{Remark}[section]

\theoremstyle{lemma}
\newtheorem{lemma}{Lemma}[section]

\theoremstyle{proposition}
\newtheorem{proposition}{Proposition}[section]

\theoremstyle{Problem}
\newtheorem{problem}{Problem}[section]

\theoremstyle{Solution}
\newtheorem{solution}{Solution}[section]

\theoremstyle{theorem}
\newtheorem{theorem}{Theorem}[section]

\theoremstyle{corollary}
\newtheorem{corollary}{Corollary}[section]

%%\input{commands.tex}

\title{DRP 2025}
\date{}
\begin{document}
\maketitle

\section{Refreshers from Lie Theory}
\begin{definition}
A representation of a Lie Algebra is a vector space $V$ with a morphism $\rho: \mathfrak{g} \to End(V)$ such that the bracket is preserved: $\rho([X,Y]) = [\rho(X),\rho(Y)]$.
\end{definition}
\begin{definition}
Given representations $V$ and $W$, the tensor representation $V\otimes W$ is given by 
$$X(v\otimes w) = (Xv)\otimes w + v\otimes (Xw)$$
\end{definition}



\section{Overview}
Our objects of study are Lie Algebras. These are vector spaces (usually over $\mathbb{C}$), equipped with a bracket. They have been fully classified in some context.
\begin{theorem}
Finite dimensional simple complex Lie algebras are classified by Dynkin Diagrams: $A_n, B_n, C_n, D_n, E_6, E_7, E_8, F_4, G_2$.
\end{theorem}
We want to understand the representation theory of these Lie Algebra. The theory is simpler in type $ADE$ (simply laced).\\

For example, the Lie algebra of type $A_n$ is $sl_{n+1}(\mathbb{C}) = \{X\in M_{n+1}(\mathbb{C}) : tr(X) = 0 \}$. The bracket is given by $[X,Y] = XY - YX$.\\
An easy representation is given by the natural action $sl_{n+1}(\mathbb{C}) \curvearrowright \mathbb{C}^{n+1}$ given by matrix operation on vectors $\rho_X(v)=Xv$. This is indeed a representation since $\rho([X,Y]) = [\rho(X),\rho(Y)]$. We can go further

\begin{proposition}
The Lie algebra $sl_{n+1}(\mathbb{C})$ acts on $\wedge^k \mathbb{C}^{n+1}$ for all $k$. Moreover, any irreducible representation arises as a subrepresentation of some tensor products of these.
\end{proposition}

Let's look precisely at $\wedge^2\mathbb{C}^4$.
First, $sl_4(\mathbb{C})$ has dimension $4^2-1 = 15$, and a Chevalley basis given by:
$$E_i = \begin{bmatrix}
1 & \delta_{1,i} & 0 & 0\\
0 & 1 & \delta_{2,i} & 0\\
0 & 0 & 1 & \delta_{2,i}\\
0 & 0 & 0 & 1
\end{bmatrix}
\quad F_i = E_i^t
\quad H_1 =  \begin{bmatrix}
1 & 0 & 0 & 0\\
0 & -1 & 0 & 0\\
0 & 0 & 0 & 0\\
0 & 0 & 0 & 0
\end{bmatrix}
\quad H_2, H_3
\quad [E_i,E_j]
\quad [F_i,F_j]
$$
The Cartan subalgebra is $\mathfrak{h} = span_\mathbb{C} (H_1, H_2, H_3)$. Note that it is commutative.\\
$\wedge^2\mathbb{C}^4$ has dimension $6$. If $e_1,e_2,e_3,e_4$ is the standard basis for $\mathbb{C}^4$, then $e_1\wedge e_2$, $e_1\wedge e_3$, $e_1\wedge e_4$, $e_2\wedge e_3$, $e_2\wedge e_4$, $e_3\wedge e_4$ forms a basis. How do the $H_i$ act on these?
$$H_1 e_1\wedge e_2 = H_1 e_1 \wedge e_2 + e_1 \wedge H_1 e_2 = 0
\quad H_2 e_1 \wedge e_2 = e_1\wedge e_2
\quad H_3 e_1\wedge e_2 = 0$$
Notice that $H_i$ acts as a scalar. Hence we make the following definition
\begin{definition}
Let $\lambda:\mathbb{C}H_1\oplus\mathbb{C}H_2\oplus \mathbb{C}H_3 \to \mathbb{C}$ be such that $H e_1\wedge e_2 = \lambda(H) e_1\wedge e_2$.
\end{definition}
Note $\lambda \in \mathfrak{h}^*$. In our case, we found out that $\lambda$ is 1 only for $H_2$ and 0 otherwise, which we write as $\lambda = \varpi_2$. In genral, $\varpi_i(H_j) = \delta_{i,j}$.





\section{The Clifford Algebra}
\begin{definition}
The tensor algebra $T^{\bullet}V = \bigoplus T^kV = \bigoplus V^{\otimes k}$ has ring structure from the maps $T^{k}V\otimes T^{k'}V \to T^{k+k'}V$.\\
Given a vector space $V$ and a quadratic bilinear form $Q$ on $V$, we define $Cliff(V,Q)$ to be the quotient of $T^{\bullet}V$ by the ideal generated by $v\otimes v - Q(v,v)1$.
\end{definition}
This algebra arises as the universal object of the following diagram:
$$\text{Given $j:V\to E$ where $E$ is any associate algebra and $i:V\to C(V,Q)$ such that $i(v)\otimes i(v) = Q(v,v)1$,}$$ 
$$\text{there exists a unique homomorphism of algebras $\phi$ such that $\phi(i(v)) = j(v)$ for all $v\in V$.}$$








\end{document}