
\documentclass[11pt]{article} 
\usepackage[left=1in, right=1in, top=1in, bottom=1in]{geometry}
\usepackage{amsmath}
\usepackage{amsthm}
\usepackage{amssymb}
\usepackage{amssymb} 
\usepackage{enumitem}
\usepackage{graphicx}
\usepackage[colorlinks=true, linkcolor=blue, citecolor=blue, urlcolor=red]{hyperref}
\usepackage{url}

%\usepackage{quiver}


%\setlist[itemize]{label=(\roman*)}

\theoremstyle{definition}
\newtheorem{definition}{Definition}[section]

\theoremstyle{example}
\newtheorem{example}{Example}[section]

\theoremstyle{remark}
\newtheorem{remark}{Remark}[section]

\theoremstyle{lemma}
\newtheorem{lemma}{Lemma}[section]

\theoremstyle{proposition}
\newtheorem{proposition}{Proposition}[section]

\theoremstyle{Problem}
\newtheorem{problem}{Problem}[section]

\theoremstyle{Solution}
\newtheorem{solution}{Solution}[section]

\theoremstyle{theorem}
\newtheorem{theorem}{Theorem}[section]

\theoremstyle{corollary}
\newtheorem{corollary}{Corollary}[section]

%%\input{commands.tex}

\title{Math 583: Geometric Group Theory}
\author{Piotr Przytycki}
\date{}
\begin{document}
\maketitle


\section{Refreshers from topology}
Perhaps surprisingly, we will study groups using results from geometry. In particular, we will look at groups by realizing them as fundamental groups. The following results will be used together repeatedly.
\begin{lemma}
Topological spaces that are homotopic have the same fundamental group. 
\end{lemma}
\begin{lemma}
Let $X$ a CW-complex, and let $A$ be a contractible subcomplex. Then, $X/A$ is homotopic to $X$. In particular, $\pi_1(X) \cong \pi_1(X/A)$.
\end{lemma}

Free groups are easily realized as topological spaces.
\begin{lemma}
The $n$-bouquet $B_n$ ($n\leq \infty$), the graph with $1$ vertex and $n$ edges, has $\pi_1(B_n) = F_n$.
\end{lemma}
\begin{proof}
Recall $\pi_1(S^1)=\mathbb{Z}$. The $n$-bouquet is the wedge product of $S^1$ with itself $n$ times. That is, gluing of circles along a single point. Since we work only up to homotopy, the choice of point does not matter.\\
The Van Kampen theorem then yields that $\pi_1(S^1\# S^1) = \mathbb{Z}*\mathbb{Z}$, and so on.
\end{proof}

\begin{proposition}
If $Y$ is a covering space of $X$, then $\pi_1(Y) < \pi_1(X)$.
\end{proposition}

One can ask if every group arises as a topological space. The answer is in the positive; we can build the so called presentation complex. An important tool is the Van Kampen Theorem, which allows to compute a fundamental group from those of a subcover.
\begin{theorem}[Van Kampen Theorem]
Let $X = X_1\cup X_2$, and $X_0 = X_1\cap X_2$. Then, $\pi_1(X) = \pi_1(X_1) \ast_{\pi_1(X_0)} \pi_1(X_2)$.
\end{theorem}

\begin{definition}[Presentation Complex]
Let $G=<A|R>$. Then, the complex $X_G$ is defined as the $|A|=n$-bouquet, with the following faces. If $R_k = a_{i_1} ... a_{i_l}$ is a relation, think of $X_g$ as a circle with $n$ points around it, all representing $x$, and edges $a_{i_1}, a_{i_2},...$ etc beteen them. Then, glue a face from point $1$ to $l$.
\end{definition}
\begin{proof}
(We define the Amalgamated Product later)
Write $X_G$ as a circle with $n$ points around it, and let $X_1$ be an interior circle and $X_2$ the outer rim (so $X_2$ is an annulus). We have $X_1$ simply connected, $\pi_1(X_2) = F_n$, and $\pi_1(X_0) = \ll R \gg$, and the rest follows by definition.\end{proof} 
Not only can every group be realized as the fundamental group of some space, but so can every homomorphism.
\begin{proposition}
Every homomorphism $G\to G'$ arises as the induced map from $f:X\to X'$ for some $X'$ with fundamental group $G'$.
\end{proposition}





\section{Prologue}
\begin{definition}[Free Product]
The free product $A*B = (S, \cdot)$ of groups is the following set and product:
\begin{itemize}
\item S is the set of all reduced words $a_1b_1a_2b_2...a_nb_n$, where $a_i\in A$ and $b_i\in B$ are all non-identity elements except maybe $a_1$ and $b_n$,
\item $w\cdot w'$ is the concatenation of words, followed by reduction (perform the possible operations inside of $A$ or $B$ that might happen where the words meet).
\end{itemize}
Note that $A$ and $B$ both live as subgroups of their free product.
\end{definition}

\begin{definition}[Free Group]
The free group $F_n$ is a free product of $\mathbb{Z}$ with itself $n$-times, where $n\leq \infty$.
\end{definition}


\begin{proposition}
Every subgroup of $F_n$ is itself free.
\end{proposition}
\begin{proof}
We have $\pi_1(B_n) = F_n$. So, subgroups of $F_n$ can be determined by looking at covering spaces of $B_n$.\\
Every covering space of $B_n$ is a graph $\Gamma$. Let $S$ a spanning tree of $\Gamma$. Since $S$ is contractible, we have $\pi_1(\Gamma) = \pi_1(\Gamma/S)$. The proof is done since $\Gamma/S$ has a single vertex, and thus is a bouquet. 
\end{proof}

The following theorem is of great importance, and is named after I. A. Gru\v{s}ko. The proof below is by Stalling.
\begin{theorem}[Gru\v{s}ko]
Let $\phi: F \to G$ is a surjective homomorphism where $F$ is free and $G = G_1 * G_2$.\\
Then, $\exists F_1, F_2 < F : \phi(F_i) = G_i$ and $F = F_1 * F_2$.
\end{theorem}
\begin{proof}
As before, we switch to complexes. Let $X_i$ be complex with $\pi_1(X_i) = G_i$, and let $X$ be the complex obtained by joining $X_1$ and $X_2$ by an edge. Place a point $x$ on this edge. Note that $\pi_1(X) = G$ by Van Kampen theorem.\\
NEXT TIME (on wednesday)...\\
At this point, we have $Y$ is a compact $2$-complex with $\pi_1(Y) = F$ so that $f:Y\to X$ is a continuous map with $f_* = \phi$. Moreover, the fiber of $x$ is a tree in $Y$.\\
We can now finish the proof. Let $F_i = f^{-1}(X_i)$, and apply Van Kampen Theorem to $Y/f^{-1}(x) = f^{-1}(X_1) \cup f^{-1}(X_2)$, noting that their intersection is $f^{-1}(x)$, a tree. We get 
$$F = \pi_1(Y) = \pi_1(Y/f^{-1}(x)) = \pi_1(f^{-1}(X_1))*\pi_1(f^{-1}(X_2)) = F_1*F_2$$
So the $F_i$ are subgroups of $F$, thus free groups by the previous theorem, and we get the advertized free product equality. \\
Note also that $f_*(F_i) = \phi(F_i) < G_i$. But since $f_*$ is surjective, and 
\end{proof}


\section{The Amalgamated Product}
In what follows, we have spaces $X_1$ and $X_2$, an open cover for $X$, with fundamental groups $G_1$ and $G_2$, and $X_0 = X_1\cap X_2$ with fundamental group $G_0$.

\begin{definition}[Amalgamated product]
The amalgamated product of groups $G_1 \ast_{G_0} G_2$ is a quotient of the free product.
The data are injections $\phi_i : G_0 \hookrightarrow G_i$. Then,
$$G_1 \ast_{G_0} G_2 = G_1 * G_2 / \ll \phi_1(g)\phi_2(g)^{-1} : \forall g \in G_0\gg$$
\end{definition}
Suppose we only know $\phi_i : G_0 \hookrightarrow G_i$, and want to build $X$ whose fundamental group is the amalgamated product. The next corollary follows directly from Van Kampen's theorem. Recall that every $\phi$ is induced by some $f$ with appropriate fundamental groups, so let $\phi_i = (f_i)_*$.
\begin{corollary}
The space $X = X_1\cup X_0 \times [0,1] \cup X_2/\sim$, with the relation $(x,0) \sim f_1(x)$ and $(x,1) \sim f_2(x)$. 
\end{corollary}
To understand the structure of the amalgamated product, we want to understand its subgroups. To do so, we look at the covers. A good place to start is with the universal cover. We first look at one side of the dumbell.
\begin{lemma}
The universal cover of $X=X_1\cup X_0\times[0,1]/\sim$ is $\widetilde{X} = \widetilde{X_1} \cup \widetilde{X_0}\times [0,1]$.
\end{lemma}
\begin{proof}

\end{proof}

\begin{proposition}
The universal cover of $X= X_1\cup X_0 \times [0,1] \cup X_2/\sim$ is 
\end{proposition}

\section{Problem Sets}
\subsection{PS1}
\begin{problem}
Let $\mu(G)$ the min cardinality of a generating set of $G$. Show $\mu(G*H) = \mu(G) + \mu(H)$.
\end{problem}
\begin{solution}
The case where either side is $\infty$ is the statement that a free product is not finitely generated iff one of the term is not, which is trivial.\\
Let $X=\{g_i\}$, $Y=\{h_i\}$ be generating sets of minimal size. Then, $X\cup Y$ is a generating set for $G*H$ by definition. $|X\cup Y| = |X| + |Y|$ so we proved $\leq$.\\
Let $T=\{t_i\}$ be any generating set for $G*H$.
\end{solution}

\begin{problem}
Let $G$ a f.g. group. Show that for some $n$, $G=G_1*...*G_n$ for $G_i$ indecomposables.
\end{problem}
\begin{solution}
Suppose $G$ is indecomposable. Then we are done. Suppose not, so $G = A*B$. Repeat the argument on $A$ and $B$, and so on. The algorithm is bound to terminate since, by problem 1, the rank of the groups strictly decrease as we go down the tree.
\end{solution}














\end{document}
