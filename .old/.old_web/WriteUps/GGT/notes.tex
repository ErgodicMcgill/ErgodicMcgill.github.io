
\documentclass[11pt]{article} 
\usepackage[left=1in, right=1in, top=1in, bottom=1in]{geometry}
\usepackage{amsmath}
\usepackage{amsthm}
\usepackage{amssymb}
\usepackage{amssymb} 
\usepackage{enumitem}
\usepackage{graphicx}
\usepackage[colorlinks=true, linkcolor=blue, citecolor=blue, urlcolor=red]{hyperref}
\usepackage{url}

%\usepackage{quiver}


%\setlist[itemize]{label=(\roman*)}

\theoremstyle{definition}
\newtheorem{definition}{Definition}[section]

\theoremstyle{example}
\newtheorem{example}{Example}[section]

\theoremstyle{remark}
\newtheorem{remark}{Remark}[section]

\theoremstyle{lemma}
\newtheorem{lemma}{Lemma}[section]

\theoremstyle{proposition}
\newtheorem{proposition}{Proposition}[section]

\theoremstyle{Problem}
\newtheorem{problem}{Problem}[section]

\theoremstyle{Solution}
\newtheorem{solution}{Solution}[section]

\theoremstyle{theorem}
\newtheorem{theorem}{Theorem}[section]

\theoremstyle{corollary}
\newtheorem{corollary}{Corollary}[section]

%%\input{commands.tex}

\title{Math 583: Geometric Group Theory}
\author{Piotr Przytycki}
\date{}
\begin{document}
\maketitle


\section{Refreshers from topology}
Perhaps surprisingly, we will study groups using results from geometry. In particular, we will look at groups by realizing them as fundamental groups. The following results will be used together repeatedly.
\begin{lemma}
Topological spaces that are homotopic have the same fundamental group. 
\end{lemma}
\begin{lemma}
Let $X$ a CW-complex, and let $A$ be a contractible subcomplex. Then, $X/A$ is homotopic to $X$. In particular, $\pi_1(X) \cong \pi_1(X/A)$.
\end{lemma}

Free groups are easily realized as topological spaces.
\begin{lemma}
The $n$-bouquet $B_n$ ($n\leq \infty$), the graph with $1$ vertex and $n$ edges, has $\pi_1(B_n) = F_n$.
\end{lemma}
\begin{proof}
Recall $\pi_1(S^1)=\mathbb{Z}$. The $n$-bouquet is the wedge product of $S^1$ with itself $n$ times. That is, gluing of circles along a single point. Since we work only up to homotopy, the choice of point does not matter.\\
The Van Kampen theorem then yields that $\pi_1(S^1\# S^1) = \mathbb{Z}*\mathbb{Z}$, and so on.
\end{proof}

\begin{proposition}
If $Y$ is a covering space of $X$, then $\pi_1(Y) < \pi_1(X)$.
\end{proposition}

One can ask if every group arises as a topological space. The answer is in the positive; we can build the so called presentation complex. An important tool is the Van Kampen Theorem, which allows to compute a fundamental group from those of a subcover.
\begin{theorem}[Van Kampen Theorem]
Let $X = X_1\cup X_2$, and $X_0 = X_1\cap X_2$. Then, $\pi_1(X) = \pi_1(X_1) \ast_{\pi_1(X_0)} \pi_1(X_2)$.
\end{theorem}

\begin{definition}[Presentation Complex]
Let $G=<A|R>$. Then, the complex $X_G$ is defined as the $|A|=n$-bouquet, with the following faces. If $R_k = a_{i_1} ... a_{i_l}$ is a relation, think of $X_g$ as a circle with $n$ points around it, all representing $x$, and edges $a_{i_1}, a_{i_2},...$ etc beteen them. Then, glue a face from point $1$ to $l$.
\end{definition}
\begin{proof}
(We define the Amalgamated Product later)
Write $X_G$ as a circle with $n$ points around it, and let $X_1$ be an interior circle and $X_2$ the outer rim (so $X_2$ is an annulus). We have $X_1$ simply connected, $\pi_1(X_2) = F_n$, and $\pi_1(X_0) = \ll R \gg$, and the rest follows by definition.\end{proof} 
Not only can every group be realized as the fundamental group of some space, but so can every homomorphism.
\begin{proposition}
Every homomorphism $G\to G'$ arises as the induced map from $f:X\to X'$ for some $X'$ with fundamental group $G'$.
\end{proposition}





\section{Prologue}
\begin{definition}[Free Product]
The free product $A*B = (S, \cdot)$ of groups is the following set and product:
\begin{itemize}
\item S is the set of all reduced words $a_1b_1a_2b_2...a_nb_n$, where $a_i\in A$ and $b_i\in B$ are all non-identity elements except maybe $a_1$ and $b_n$,
\item $w\cdot w'$ is the concatenation of words, followed by reduction (perform the possible operations inside of $A$ or $B$ that might happen where the words meet).
\end{itemize}
Note that $A$ and $B$ both live as subgroups of their free product.
\end{definition}

\begin{definition}[Free Group]
The free group $F_n$ is a free product of $\mathbb{Z}$ with itself $n$-times, where $n\leq \infty$.
\end{definition}


\begin{proposition}
Every subgroup of $F_n$ is itself free.
\end{proposition}
\begin{proof}
We have $\pi_1(B_n) = F_n$. So, subgroups of $F_n$ can be determined by looking at covering spaces of $B_n$.\\
Every covering space of $B_n$ is a graph $\Gamma$. Let $S$ a spanning tree of $\Gamma$. Since $S$ is contractible, we have $\pi_1(\Gamma) = \pi_1(\Gamma/S)$. The proof is done since $\Gamma/S$ has a single vertex, and thus is a bouquet. 
\end{proof}

The following theorem is of great importance, and is named after I. A. Gru\v{s}ko. The proof below is by Stalling.
\begin{theorem}[Gru\v{s}ko]
Let $\phi: F \to G$ is a surjective homomorphism where $F$ is free and $G = G_1 * G_2$.\\
Then, $\exists F_1, F_2 < F : \phi(F_i) = G_i$ and $F = F_1 * F_2$.
\end{theorem}
\begin{proof}
As before, we switch to complexes. Let $X_i$ be complex with $\pi_1(X_i) = G_i$, and let $X$ be the complex obtained by joining $X_1$ and $X_2$ by an edge. Place a point $x$ on this edge. Note that $\pi_1(X) = G$ by Van Kampen theorem.


We can find $Y$ compact $2$-complex with $\pi_1(Y) = F$ so that $f:Y\to X$ is a continuous map with $f_* = \phi$. Our goal is to show that the fiber of $x$ is a tree in $Y$.

We can now finish the proof. Let $F_i = f^{-1}(X_i)$, and apply Van Kampen Theorem to $Y/f^{-1}(x) = f^{-1}(X_1) \cup f^{-1}(X_2)$, noting that their intersection is $f^{-1}(x)$, a tree. We get 
$$F = \pi_1(Y) = \pi_1(Y/f^{-1}(x)) = \pi_1(f^{-1}(X_1))*\pi_1(f^{-1}(X_2)) = F_1*F_2$$
So the $F_i$ are subgroups of $F$, thus free groups by the previous theorem, and we get the advertized free product equality. \\
Note also that $f_*(F_i) = \phi(F_i) < G_i$. But since $f_*$ is surjective we must have equality. 


To find the objects in our goal, we do the following. In general, $f^{-1}(x)$ is a forest. We decrease the number of components by finding a path $l\in Y^{(1)}$ connecting different trees, and so that $f(l)$ falls to a trivial element of $\pi_1(X_i)$. If we can do that, then attach an edge $e$ to the ends of $l$ and a cell $D$ delimited by $e\cup l$. Since $f(l)$ is trivial, we can extend $f$ so that $f^{-1}(x)$ now contains $e$.

To find $l$, start with an edge path $L$ joining trees in $f^{-1}(x)$. By surjectivity of $\phi=f_*$, we can find $\gamma$ a closed edge-path in $Y$ based at the starting point of $L$ with $[f(\gamma)] = [f(L)]$ in $\pi_1(X)$. Let $l=\gamma^{-1}L$. This is a path with $f^{-1}(l)$ trivial in $\pi_1(X)$. Patrtition $l=l_1l_2...l_n$ where all the endpoints aree in $f^{-1}(x)$, and $l_i$ is mapped alternatingly in $X_1$ and $X_2$. Let $g_j=[f(l_j)]$. If $g_j=1$, but then endpoints of $l_j$ lie on the same tree in $f^{-1}(x)$, we can replace $l_j$ by a path in $f^{-1}(x)$ which we merge with $l_{j-1}$ and $l_{j+1}$. At this point, $[f(l)] = 1 = g_1....g_n$ lying alternatingly in $G_1$ and $G_2$. So at least one $g_j$ is trivial, and $l_j$ is the desired path.
\end{proof}


\section{The Amalgamated Product}
In what follows, we have spaces $X_1$ and $X_2$, an open cover for $X$, with fundamental groups $G_1$ and $G_2$, and $X_0 = X_1\cap X_2$ with fundamental group $G_0$.

\begin{definition}[Amalgamated product]
The amalgamated product of groups $G_1 \ast_{G_0} G_2$ is a quotient of the free product.
The data are injections $\phi_i : G_0 \hookrightarrow G_i$. Then,
$$G_1 \ast_{G_0} G_2 = G_1 * G_2 / \ll \phi_1(g)\phi_2(g)^{-1} : \forall g \in G_0\gg$$
\end{definition}
Suppose we only know $\phi_i : G_0 \hookrightarrow G_i$, and want to build $X$ whose fundamental group is the amalgamated product. The next corollary follows directly from Van Kampen's theorem. Recall that every $\phi$ is induced by some $f$ with appropriate fundamental groups, so let $\phi_i = (f_i)_*$.
\begin{corollary}
The space $X = X_1\cup X_0 \times [0,1] \cup X_2/\sim$, with the relation $(x,0) \sim f_1(x)$ and $(x,1) \sim f_2(x)$. 
\end{corollary}
To understand the structure of the amalgamated product, we want to understand its subgroups. To do so, we look at the covers. A good place to start is with the universal cover. We first look at one side of the dumbell.
\begin{lemma}
The universal cover of $Y_1=X_1\cup X_0\times[0,1]/\sim$ is $\widetilde{Y_1} = \widetilde{X_1} \cup \bigsqcup\widetilde{X_0}\times [0,1]/\sim$.
\end{lemma}
\begin{proof}
Let $p:\widetilde{Y_i}\to Y_i$ be the universal cover. First, we show $\widetilde{X_0}\times [0,1]$ are the components of $p^{-1}(X_0\times[0,1])$. If not, then there is $\alpha$ some loop in $X_0$ giving a nontrivial element in $\pi_1(X_0)$, and it lifts to a loop $\tilde{\alpha}$ in $\widetilde{X_0}\times [0,1]$. Since $\phi^i$ is injective (by the data), $\phi^i(\alpha)$ is a non-trivial element of $\pi_1(X_1)$, so it's lift $\tilde{alpha}$ must have disjoint endopoints, contradicting that $\tilde{\alpha}$ was a loop.

Then, $\widetilde{Y_1}$ is a copy of $\widetilde{X_1}$ with copys of $\widetilde{X_0}\times [0,1]$ glued via the lifts $\widetilde{X_0}\to\widetilde{X_i}$ of $f^i$.
\end{proof}
	
\begin{proposition}
The universal cover of $X= X_1\cup X_0 \times [0,1] \cup X_2/\sim$ is copies of $\widetilde{Y_1}$ and $\widetilde{Y_2}$ glued in a tree like fashion by cylinders $\widetilde{X_0}$. 
\end{proposition}
\begin{proof}
This construction is simply connected.
\end{proof}


\begin{corollary}[Normal Form]
Let $G=G_1 \ast_{G_0}G_2$. For all $g\in G\backslash G_0$, $g=g_1h_1g_2h_2...g_nh_n$. Moreover, this expression is unique up to inserting $g_0g_0^{-1}$ in the following way: $g_ih_i = (g_ig_0)(g_0^{-1}h_i)$.
\end{corollary}


\subsection{Bass-Serre Theory}

\begin{definition}
The Bass-Serre tree is the tree obtained by collapsing $\widetilde{X_i}$ to vertices and $\widetilde{X_0}\times[0,1]$ to edges in the fundamental cover of $X$.
\end{definition}

\begin{corollary}
The group $G$ acts on it's Bass-Serre tree.
\end{corollary}

The Bass-Serre tree $\Gamma=(V,E)$ can be defined purely algebraically as follows. Recall the data: $G=G_1 \ast_{G_0}G_2$ (in particular, injections $G_0 \hookrightarrow G_i$). This description will allow us to say things about the graph.
\begin{itemize}
\item V is the set of cosets $gG_1$, $gG_2$ for $g\in G$.
\item Edges are cosets $gG_0$ for $g\in G$.
\end{itemize}
\begin{proposition}
The degree of a vertex $v=gG_i$ is $[G_i: G_0]$ (note that it does not depend on $g$).
The action of $h\in G$ on a vertex $v=gG_i$ is $h(v)=h(gG_i)=(hg)G_i$, and the action on edges is induced.
\end{proposition}

\begin{example}
\end{example}

The Bass-Serre tree can be endowed with extra structure: it is a graph of groups.
\begin{definition}
A graph of group $\mathfrak{G}$ is three things:
\begin{itemize}
\item A graph $\Gamma = (V,E)$.
\item A collection of groups $G_v$ for all vertices, and $G_e$ for all edges,
\item For all edge $e=(v,w)$, injective homomorphisms $\phi_e^v:G_e \to G_v$ and $\phi_e^w:G_e \to G_w$.
\end{itemize}
\end{definition}
In a similar fashion, we can define a graph of spaces:
\begin{definition}
A graph of spaces $\mathfrak{X}$ is three things:
\begin{itemize}
\item A graph $\Gamma = (V,E)$.
\item A collection of spaces $X_v$ for all vertices, 
\item For all edge $e=(v,w)$, injective maps $f_e^v:G_e \to G_v$ and $f_e^w:G_e \to G_w$.
\end{itemize}
\end{definition}










\end{document}
