\documentclass[reqno, twoside]{article}
\input{preamble.sty}
\input{macros.sty}

\begin{document}
    \title{\textbf{\normalsize\MakeUppercase{Summer 2025 Reading Group on Ergodic Theory}}}
    \author{\normalsize\textsc{Exercise Sheet 5 (Zhaoshen Zhai): Applications and Generalizations}\\ \normalsize\textsc{of Birkhoff's Pointwise Ergodic Theorem}}
    \date{}
    \maketitle
    \freefootnote{\textit{Date}: \today.}

    Throughout, let $(X,\mu,T)$ be a measure-preserving dynamical system. The purpose of this exercise sheet is to give some applications and generalizations of the Pointwise Ergodic Theorem, which for convenience, we provide a sketch here. For $f\in L^1(X,\mu)$ and $n\in\N$, let $A_n^Tf\coloneqq\frac{1}{n}\sum_{i<n}f\circ T^i$.

    \begin{theorem*}
        If $T$ is ergodic, then for any $f\in L^1(X,\mu)$, we have $\lim_nA_n^Tf=_\mu\int f\,\d\mu$.
    \end{theorem*}
    \begin{proof}[Proof sketch]
        Assume $\int f\,\d\mu=0$ and recall that $l\coloneqq\limsup_nA_n^Tf:X\to\R$ is $T$-invariant, so $l$ is constant $\mu$-a.e. by ergodicity, say at $l_0\in\R$. Suppose that $f^\ast\coloneqq l_0/2>0$, so for each $x\in X$, there is a minimal $\eta(x)\in\N$ such that $A^T_{\eta(x)}f(x)\geq f^\ast$. We are done if there is a uniform $n\in\N$ such that $A_n^Tf\geq f^\ast/2$. This is not true in general, but after trimming measure-$\epsilon$ parts of $X$, something like this can be done using:
        \begin{center}
            \begin{minipage}{0.95\textwidth}
                \begin{lemma*}[Tiling Lemma]
                    Let $\eta:X\to\N$ be an arbitrary measurable function. For any $\epsilon>0$, there exists $n\gg0$ such that for each $x\in X$ except on a measure-$\epsilon$ set, the interval $I_n^T(x)$ can be tiled, up to an $\epsilon$-fraction, by intervals of the form $I_y\coloneqq I_{\eta(y)}^T(y)$ for $y\in X$.
                \end{lemma*}
            \end{minipage}
        \end{center}\vspace{-0.25in}\qed
    \end{proof}

    \begin{exercise}
        In the above context, prove that if both $f$ and $\eta$ are bounded, then $A_n^Tf\geq f^\ast/2$.

        \textsc{Hint}: Prove a stronger Tiling Lemma in this case.
    \end{exercise}

    \begin{exercise}
        What is the average value of a given digit $0\leq m\leq9$, say $m\coloneqq7$, to occur in the decimal representation of $\lambda$-a.e. $x\in[0,1]$? That is, does $\lim_n\frac{1}{n}|\{i<n\st x_i=m\}|$ exist, and what is it?

        \textsc{Hint:} Consider the $10$-ary Baker's map $b_{10}:[0,1)\to[0,1)$  sending $x\mapsto 10x$ (mod $1$), which is isomorphic to the shift map on $10^\N$.
    \end{exercise}

    \begin{exercise}[Equidistribution Theorem]
        A sequence $(x_n)_n$ in $S^1$ is said to be \textit{equidistributed} if for every interval $I\subeq S^1$, we have $\lim_n\frac{1}{n}|\l\{x_i\r\}_{i<n}\cap I|=\lambda(I)$. Prove that if $x_n=n\alpha$ for some irrational $\alpha\in S^1$, then $(x_n)_n$ is equidistributed. \textsc{Hint:} Don't overthink it.
    \end{exercise}

    \begin{exercise}[Law of Large Numbers]
        If you know statistics, prove it! If not, skip it.
    \end{exercise}

    \begin{exercise}[An ergodic theorem for non-ergodic actions]
        Intuitively, Birkhoff's Pointwise Ergodic Theorem states that ergodic transformations $T:X\to X$ stir up $X$ so well that they spread any $f\in L^1(X,\mu)$ evenly on $X$, making it constant at $\int f\,\d\mu$; indeed, `$f\circ T^\infty=\int f\,\d\mu$'.

        If $T$ is not ergodic, then there is a non-trivial partition $X=X_1\sqcup X_2$ into $T$-invariant pieces. The best that one can hope is at after `enough' partitions $X=\bigsqcup_iX_i$, $T$ still spreads each $f_i\coloneqq f\chi_{X_i}$ evenly on $X_i$. Viewing $f$ from the lens of these $T$-invariant pieces leads to the \textit{conditional expectation} of $f$:
        \begin{center}
            \begin{minipage}{0.95\textwidth}
                \begin{definition*}
                    Let $\mc{A}\subeq\mc{B}(X)$ be a sub-$\sigma$-algebra of $\mc{B}(X)$. For each $f\in L^1(X,\mu)$, there is a unique (up to a $\mu$-null set) $\mc{A}$-measurable function $f_\mc{A}$ such that $\int_Af\,\d\mu=\int_Af_\mc{A}\,\d\mu$ for each $A\in\mc{A}$, called the \textit{conditional expectation of $f$ w.r.t. $\mc{A}$}. We write $\E(f|\mc{A})$ for $f_\mc{A}$.
                \end{definition*}
                \vspace{-0.20in}
                \begin{remark*}
                    If $\mc{P}\subeq\mc{B}(X)$ is a countable partition of $X$, then $\E(f|\gen{\mc{P}}_\sigma)=\sum_{P\in\mc{P}}\l(\frac{1}{\mu(P)}\int_Pf\,\d\mu\r)\chi_P$.
                \end{remark*}
                \vspace{-0.20in}
            \end{minipage}
        \end{center}
        Prove that for any (not necessarily ergodic) pmp transformation $T:X\to X$ and any $f\in L^1(X,\mu)$, we have $\lim_nA_n^Tf=_\mu\E(f|\mc{B}_T)$, where $\mc{B}_T\subeq\mc{B}(X)$ is the $\sigma$-algebra generated by all $T$-invariant Borel sets of $X$.

        \textsc{Hint}: Same as the regular proof, only that $f^\ast:X\to\R$ is not necessarily constant, but just $T$-invariant.
    \end{exercise}

    \begin{exercise}[$L^p$-ergodic theorem]
        Prove that for any $p\geq1$, we have $A_n^Tf\to_{L^p}\E(f|\mc{B}_T)$ for all $f\in L^p(X,\mu)$.

        \textsc{Hint}: If $f$ is bounded, then we are done by the DCT. Otherwise, let $f_k\to_{L^p}f$ where each $f_k$ is bounded and triangle-inequality your way through, using that $\|A_n^Tf\|_{L^p}\leq\|f\|_{L^p}$ (prove this too).
    \end{exercise}
\end{document}
