\documentclass[reqno, twoside]{article}
\usepackage{amsfonts, amsmath, amssymb, amsthm}
\usepackage{mathtools, mathrsfs, dsfont, nicematrix}
\usepackage{tikz, tikz-3dplot, tikzpagenodes, graphicx, xcolor, mlmodern}
\usepackage{geometry, mdframed, titlesec, fancyhdr, caption, subcaption}
\usepackage{enumitem, bookmark, xifthen, setspace, proof}

\usetikzlibrary{matrix, positioning, patterns, decorations.markings, arrows, arrows.meta, backgrounds, math, cd}
\tikzset{->-/.style={decoration={ markings, mark=at position #1 with {\arrow{>}}},postaction={decorate}}}

\definecolor{darkBlue}{RGB}{0, 0, 138}
\definecolor{darkGreen}{RGB}{0, 160, 0}
\definecolor{lightGray}{RGB}{128, 128, 128}
\hypersetup{colorlinks=true, allcolors=darkBlue}

\newgeometry{margin = 1in}
\pagestyle{fancy}\fancyhead{}\fancyfoot{}\headsep=15pt\renewcommand{\headrulewidth}{0pt}
\fancyhead[RO, LE]{\scriptsize\thepage}
\fancyhead[CE]{\scriptsize\MakeUppercase{Zhaoshen Zhai}}
\fancyhead[CO]{\scriptsize\MakeUppercase{Reading group on Ergodic Theory}}

\titleformat{name=\section}{}{\thetitle.}{0.8em}{\centering\scshape}
\titleformat{name=\subsection}[runin]{}{\thetitle.}{0.5em}{\bfseries}[.]
\titleformat{name=\subsubsection}[runin]{}{\thetitle.}{0.5em}{\itshape}[.]

\usepackage[style=alphabetic-verb, sorting=nty]{biblatex}
\addbibresource{refs.bib}

\newtheorem{mainTheorem}{Theorem}\renewcommand{\themainTheorem}{\Alph{mainTheorem}}
\newtheorem{theorem}{Theorem}[section]\newtheorem*{theorem*}{Theorem}
\newtheorem{proposition}[theorem]{Proposition}\newtheorem*{proposition*}{Proposition}
\newtheorem{lemma}[theorem]{Lemma}\newtheorem*{lemma*}{Lemma}
\newtheorem{claim}[theorem]{Claim}\newtheorem*{claim*}{Claim}
\newtheorem{thesis}[theorem]{Thesis}\newtheorem*{thesis*}{Thesis}
\newtheorem{corollary}[theorem]{Corollary}\newtheorem*{corollary*}{Corollary}

\theoremstyle{definition}{
    \newtheorem{example}[theorem]{Example}\newtheorem*{example*}{Example}
    \newtheorem{definition}[theorem]{Definition}\newtheorem*{definition*}{Definition}
    \newtheorem{remark}[theorem]{Remark}\newtheorem*{remark*}{Remark}
    \newtheorem{notation}[theorem]{Notation}\newtheorem*{notation*}{Notation}
    \newtheorem{observation}[theorem]{Observation}\newtheorem*{observation*}{Observation}
    \newtheorem{axiom}[theorem]{Axiom}\newtheorem*{axiom*}{Axiom}
    \newtheorem{question}[theorem]{Question}\newtheorem*{question*}{Question}
    \newtheorem{exercise}[theorem]{Exercise}\newtheorem*{exercise*}{Exercise}
    \newtheorem{fact}[theorem]{Fact}\newtheorem*{fact*}{Fact}
    \newtheorem{open}[theorem]{Open Question}\newtheorem*{open*}{Open Question}
}

\newmdenv[topline=false, bottomline=false, rightline=false, skipabove=4pt, skipbelow=4pt, linewidth=0.75pt]{leftbar}
\newcommand\freefootnote[1]{{\renewcommand\thefootnote{}\footnote{#1}\addtocounter{footnote}{-1}}}

% Operators
    \newcommand{\id}{\operatorname{id}}
    \newcommand{\im}{\operatorname{im}}
    \newcommand{\rk}{\operatorname{rk}}
    \newcommand{\ch}{\operatorname{ch}}
    \newcommand{\tr}{\operatorname{tr}}
    \newcommand{\tp}{\operatorname{tp}}
    \newcommand{\ot}{\operatorname{ot}}
    \newcommand{\ad}{\operatorname{ad}}
    \newcommand{\cf}{\operatorname{cf}}
    \newcommand{\Id}{\operatorname{Id}}
    \newcommand{\Th}{\operatorname{Th}}
    \newcommand{\Cn}{\operatorname{Cn}}
    \newcommand{\Bl}{\operatorname{Bl}}
    \newcommand{\Cl}{\operatorname{Cl}}
    \newcommand{\Ad}{\operatorname{Ad}}
    \newcommand{\LT}{\operatorname{LT}}
    \newcommand{\dom}{\operatorname{dom}}
    \newcommand{\ran}{\operatorname{ran}}
    \newcommand{\cdm}{\operatorname{cdm}}
    \newcommand{\sgn}{\operatorname{sgn}}
    \newcommand{\lcm}{\operatorname{lcm}}
    \newcommand{\ord}{\operatorname{ord}}
    \newcommand{\cvx}{\operatorname{cvx}}
    \newcommand{\Aut}{\operatorname{Aut}}
    \newcommand{\Inn}{\operatorname{Inn}}
    \newcommand{\Out}{\operatorname{Out}}
    \newcommand{\End}{\operatorname{End}}
    \newcommand{\Mat}{\operatorname{Mat}}
    \newcommand{\Obj}{\operatorname{Obj}}
    \newcommand{\Hom}{\operatorname{Hom}}
    \newcommand{\Tor}{\operatorname{Tor}}
    \newcommand{\Ext}{\operatorname{Ext}}
    \newcommand{\Ann}{\operatorname{Ann}}
    \newcommand{\Sym}{\operatorname{Sym}}
    \newcommand{\Alt}{\operatorname{Alt}}
    \newcommand{\Cov}{\operatorname{Cov}}
    \newcommand{\Orb}{\operatorname{Orb}}
    \newcommand{\Sat}{\operatorname{Sat}}
    \newcommand{\Thm}{\operatorname{Thm}}
    \newcommand{\Der}{\operatorname{Der}}
    \newcommand{\Def}{\operatorname{Def}}
    \newcommand{\Age}{\operatorname{Age}}
    \newcommand{\Div}{\operatorname{Div}}
    \newcommand{\Lie}{\operatorname{Lie}}
    \newcommand{\Rep}{\operatorname{Rep}}
    \newcommand{\Bil}{\operatorname{Bil}}
    \newcommand{\Ind}{\operatorname{Ind}}
    \newcommand{\Res}{\operatorname{Res}}
    \newcommand{\Cay}{\operatorname{Cay}}
    \newcommand{\Min}{\operatorname{Min}}
    \newcommand{\Con}{\operatorname{Con}}
    \newcommand{\PGL}{\operatorname{PGL}}
    \newcommand{\MPT}{\operatorname{MPT}}
    \newcommand{\rank}{\operatorname{rank}}
    \newcommand{\proj}{\operatorname{proj}}
    \newcommand{\diag}{\operatorname{diag}}
    \newcommand{\eval}{\operatorname{eval}}
    \newcommand{\cont}{\operatorname{cont}}
    \newcommand{\diam}{\operatorname{diam}}
    \newcommand{\mult}{\operatorname{mult}}
    \newcommand{\trcl}{\operatorname{trcl}}
    \newcommand{\supp}{\operatorname{supp}}
    \newcommand{\conv}{\operatorname{conv}}
    \newcommand{\Core}{\operatorname{Core}}
    \newcommand{\Term}{\operatorname{Term}}
    \newcommand{\Taut}{\operatorname{Taut}}
    \newcommand{\Sent}{\operatorname{Sent}}
    \newcommand{\Skew}{\operatorname{Skew}}
    \newcommand{\Frac}{\operatorname{Frac}}
    \newcommand{\Stab}{\operatorname{Stab}}
    \newcommand{\Null}{\operatorname{Null}}
    \newcommand{\Meas}{\operatorname{Meas}}
    \newcommand{\Diag}{\operatorname{Diag}}
    \newcommand{\Diff}{\operatorname{Diff}}
    \newcommand{\Isom}{\operatorname{Isom}}
    \newcommand{\Area}{\operatorname{Area}}
    \newcommand{\coker}{\operatorname{coker}}
    \newcommand{\preim}{\operatorname{preim}}
    \newcommand{\Graph}{\operatorname{Graph}}
    \newcommand{\Axioms}{\operatorname{Axioms}}
    \renewcommand{\Re}{\operatorname{Re}}
    \renewcommand{\Im}{\operatorname{Im}}

% Math notations
    % Set Theory, Category Theory, and Logic
        \newcommand{\e}{\varnothing}
        \newcommand{\fa}{\forall}
        \newcommand{\ex}{\exists}
        \newcommand{\iso}{\cong}
        \newcommand{\cat}[1]{\mathbf{#1}}
        \newcommand{\pow}{\mathcal{P}}
        \newcommand{\comp}{\setminus}
        \newcommand{\limp}{\multimap}
        \newcommand{\code}[2][]{\ulcorner#1#2#1\urcorner}
        \newcommand{\cprod}{\amalg}
        \newcommand{\eqnum}{\approx}
        \newcommand{\natiso}{\simeq}
        \newcommand{\proves}{\vdash}
        \newcommand{\forces}{\Vdash}
        \newcommand{\nproves}{\nvdash}
        \newcommand{\symdiff}{\bigtriangleup}
        \newcommand{\ladjoint}{\dashv}
        \newcommand{\radjoint}{\vdash}
        \renewcommand{\vec}[1]{\bar{#1}}

    % Analysis
        \newcommand{\BV}{BV}
        \newcommand{\del}{\partial}
        \newcommand{\abscont}{\ll}
        \newcommand{\esssup}{\operatorname{ess-sup}}
        \renewcommand{\d}{\mathrm{d}}

    % Topology
        \newcommand{\htopeq}{\simeq}

    % Group Theory
        \newcommand{\act}{\curvearrowright}
        \newcommand{\acted}{\curvearrowleft}
        \newcommand{\semiact}{\ltimes}
        \newcommand{\semiacted}{\rtimes}

    % Number Theory
        \newcommand{\ndiv}{\nmid}
        \renewcommand{\div}{\,|\,}

    % Misc
        \newcommand{\st}{:}
        \newcommand{\tpl}[1]{\l(#1\r)}
        \newcommand{\gen}[2][]{\ifthenelse{\isempty{#1}}{}{\l}\langle#2\ifthenelse{\isempty{#1}}{}{\r}\rangle}
        \newcommand{\slot}{-}
        \newcommand{\blob}{\bullet}
        \renewcommand{\l}{\left}
        \renewcommand{\r}{\right}
        \renewcommand{\bar}{\overline}

    % Arrows
        \newcommand{\too}[2][]{\xlongrightarrow[#1]{#2}}
        \newcommand{\onto}{\twoheadrightarrow}
        \newcommand{\into}{\hookrightarrow}
        \newcommand{\intoo}[2][]{\lhook\joinrel\xlongrightarrow[#1]{#2}}
        \newcommand{\ontoo}[2][]{\xlongrightarrow[#1]{#2}\mathrel{\mkern-14mu}\rightarrow}
        \newcommand{\parto}{\rightharpoonup}
        \newcommand{\ratto}{\dashrightarrow}
        \newcommand{\incto}{\nearrow}
        \newcommand{\decto}{\searrow}
        \newcommand{\pathto}{\rightsquigarrow}

    % Subobjects
        \newcommand{\sub}{\subset}
        \newcommand{\subs}{<}
        \newcommand{\sups}{>}
        \newcommand{\esub}{\prec}
        \newcommand{\esup}{\succ}
        \newcommand{\nsub}{\triangleleft}
        \newcommand{\nsup}{\triangleright}
        \newcommand{\subeq}{\subseteq}
        \newcommand{\subseq}{\leq}
        \newcommand{\supseq}{\geq}
        \newcommand{\esubeq}{\preceq}
        \newcommand{\esupeq}{\succeq}
        \newcommand{\nsubeq}{\trianglelefteq}
        \newcommand{\nsupeq}{\trianglerighteq}

    % Number Systems
        \newcommand{\N}{\mathbb{N}}
        \newcommand{\Z}{\mathbb{Z}}
        \newcommand{\Q}{\mathbb{Q}}
        \newcommand{\R}{\mathbb{R}}
        \newcommand{\C}{\mathbb{C}}
        \newcommand{\F}{\mathbb{F}}
        \newcommand{\E}{\mathbb{E}}
        \newcommand{\A}{\mathbb{A}}
        \renewcommand{\H}{\mathbb{H}}
        \renewcommand{\S}{\mathbb{S}}
        \renewcommand{\P}{\mathbb{P}}

% LaTeX
    % Fonts
        \newcommand{\mc}[1]{\mathcal{#1}}
        \newcommand{\ms}[1]{\mathscr{#1}}
        \newcommand{\mb}[1]{\mathbb{#1}}
        \newcommand{\mf}[1]{\mathfrak{#1}}
        \newcommand{\ds}[1]{\mathds{#1}}
        \newcommand{\bb}[1]{\mathbb{#1}}
        \newcommand{\mathsc}[1]{{\normalfont\textsc{#1}}}
        \renewcommand{\it}[1]{\textit{#1}}
        \renewcommand{\bf}[1]{\textbf{#1}}
        \renewcommand{\sf}[1]{\textsf{#1}}
        \renewcommand{\phi}{\varphi}
        \renewcommand{\epsilon}{\varepsilon}

    % Meta
        \newcommand{\TODO}[1][]{{\color{red}\textbf{TODO}\ifthenelse{\isempty{#1}}{}{\textbf{:} #1}}}
        \newcommand{\qedin}{\tag*{$\blacksquare$}}
        \newcommand{\qedlemin}{\tag*{$\square$}}
        \newcommand{\qedlem}{\phantom\qedhere\hfill$\square$}
        \renewcommand{\qed}{\phantom\qedhere\hfill$\blacksquare$}


\begin{document}
    \title{\textbf{\normalsize\MakeUppercase{Summer 2025 Reading Group on Ergodic Theory}}}
    \author{\normalsize\textsc{Exercise Sheet 5 (Zhaoshen Zhai): Applications and Generalizations}\\ \normalsize\textsc{of Birkhoff's Pointwise Ergodic Theorem}}
    \date{}
    \maketitle
    \freefootnote{\textit{Date}: \today.}

    Throughout, let $(X,\mu,T)$ be a measure-preserving dynamical system. The purpose of this exercise sheet is to give some applications and generalizations of the Pointwise Ergodic Theorem, which for convenience, we provide a sketch here. For $f\in L^1(X,\mu)$ and $n\in\N$, let $A_n^Tf\coloneqq\frac{1}{n}\sum_{i<n}f\circ T^i$.

    \begin{theorem*}
        If $T$ is ergodic, then for any $f\in L^1(X,\mu)$, we have $\lim_nA_n^Tf=_\mu\int f\,\d\mu$.
    \end{theorem*}
    \begin{proof}[Proof sketch]
        Assume $\int f\,\d\mu=0$ and recall that $l\coloneqq\limsup_nA_n^Tf:X\to\R$ is $T$-invariant, so $l$ is constant $\mu$-a.e. by ergodicity, say at $l_0\in\R$. Suppose that $f^\ast\coloneqq l_0/2>0$, so for each $x\in X$, there is a minimal $\eta(x)\in\N$ such that $A^T_{\eta(x)}f(x)\geq f^\ast$. We are done if there is a uniform $n\in\N$ such that $A_n^Tf\geq f^\ast/2$. This is not true in general, but after trimming measure-$\epsilon$ parts of $X$, something like this can be done using:
        \begin{center}
            \begin{minipage}{0.95\textwidth}
                \begin{lemma*}[Tiling Lemma]
                    Let $\eta:X\to\N$ be an arbitrary measurable function. For any $\epsilon>0$, there exists $n\gg0$ such that for each $x\in X$ except on a measure-$\epsilon$ set, the interval $I_n^T(x)$ can be tiled, up to an $\epsilon$-fraction, by intervals of the form $I_y\coloneqq I_{\eta(y)}^T(y)$ for $y\in X$.
                \end{lemma*}
            \end{minipage}
        \end{center}\vspace{-0.25in}\qed
    \end{proof}

    \begin{exercise}
        In the above context, prove that if both $f$ and $\eta$ are bounded, then $A_n^Tf\geq f^\ast/2$.

        \textsc{Hint}: Prove a stronger Tiling Lemma in this case.
    \end{exercise}

    \begin{exercise}
        What is the average value of a given digit $0\leq m\leq9$, say $m\coloneqq7$, to occur in the decimal representation of $\lambda$-a.e. $x\in[0,1]$? That is, does $\lim_n\frac{1}{n}|\{i<n\st x_i=m\}|$ exist, and what is it?

        \textsc{Hint:} Consider the $10$-ary Baker's map $b_{10}:[0,1)\to[0,1)$  sending $x\mapsto 10x$ (mod $1$), which is isomorphic to the shift map on $10^\N$.
    \end{exercise}

    \begin{exercise}[Equidistribution Theorem]
        A sequence $(x_n)_n$ in $S^1$ is said to be \textit{equidistributed} if for every interval $I\subeq S^1$, we have $\lim_n\frac{1}{n}|\l\{x_i\r\}_{i<n}\cap I|=\lambda(I)$. Prove that if $x_n=n\alpha$ for some irrational $\alpha\in S^1$, then $(x_n)_n$ is equidistributed. \textsc{Hint:} Don't overthink it.
    \end{exercise}

    \begin{exercise}[Law of Large Numbers]
        If you know statistics, prove it! If not, skip it.
    \end{exercise}

    \begin{exercise}[An ergodic theorem for non-ergodic actions]
        Intuitively, Birkhoff's Pointwise Ergodic Theorem states that ergodic transformations $T:X\to X$ stir up $X$ so well that they spread any $f\in L^1(X,\mu)$ evenly on $X$, making it constant at $\int f\,\d\mu$; indeed, `$f\circ T^\infty=\int f\,\d\mu$'.

        If $T$ is not ergodic, then there is a non-trivial partition $X=X_1\sqcup X_2$ into $T$-invariant pieces. The best that one can hope is at after `enough' partitions $X=\bigsqcup_iX_i$, $T$ still spreads each $f_i\coloneqq f\chi_{X_i}$ evenly on $X_i$. Viewing $f$ from the lens of these $T$-invariant pieces leads to the \textit{conditional expectation} of $f$:
        \begin{center}
            \begin{minipage}{0.95\textwidth}
                \begin{definition*}
                    Let $\mc{A}\subeq\mc{B}(X)$ be a sub-$\sigma$-algebra of $\mc{B}(X)$. For each $f\in L^1(X,\mu)$, there is a unique (up to a $\mu$-null set) $\mc{A}$-measurable function $f_\mc{A}$ such that $\int_Af\,\d\mu=\int_Af_\mc{A}\,\d\mu$ for each $A\in\mc{A}$, called the \textit{conditional expectation of $f$ w.r.t. $\mc{A}$}. We write $\E(f|\mc{A})$ for $f_\mc{A}$.
                \end{definition*}
                \vspace{-0.20in}
                \begin{remark*}
                    If $\mc{P}\subeq\mc{B}(X)$ is a countable partition of $X$, then $\E(f|\gen{\mc{P}}_\sigma)=\sum_{P\in\mc{P}}\l(\frac{1}{\mu(P)}\int_Pf\,\d\mu\r)\chi_P$.
                \end{remark*}
                \vspace{-0.20in}
            \end{minipage}
        \end{center}
        Prove that for any (not necessarily ergodic) pmp transformation $T:X\to X$ and any $f\in L^1(X,\mu)$, we have $\lim_nA_n^Tf=_\mu\E(f|\mc{B}_T)$, where $\mc{B}_T\subeq\mc{B}(X)$ is the $\sigma$-algebra generated by all $T$-invariant Borel sets of $X$.

        \textsc{Hint}: Same as the regular proof, only that $f^\ast:X\to\R$ is not necessarily constant, but just $T$-invariant.
    \end{exercise}

    \begin{exercise}[$L^p$-ergodic theorem]
        Prove that for any $p\geq1$, we have $A_n^Tf\to_{L^p}\E(f|\mc{B}_T)$ for all $f\in L^p(X,\mu)$.

        \textsc{Hint}: If $f$ is bounded, then we are done by the DCT. Otherwise, let $f_k\to_{L^p}f$ where each $f_k$ is bounded and triangle-inequality your way through, using that $\|A_n^Tf\|_{L^p}\leq\|f\|_{L^p}$ (prove this too).
    \end{exercise}
\end{document}
